\documentclass[11pt,a4paper]{scrartcl}
\usepackage[utf8]{inputenc}
\usepackage{amsmath}
\usepackage{amsfonts}
\usepackage{pgfpages}
\usepackage{amssymb}
\usepackage[hidelinks]{hyperref}
\usepackage[utf8]{inputenc}
\usepackage[ngerman]{babel}
\usepackage[autostyle]{csquotes}
\usepackage{eurosym}
\usepackage[left=2.54cm,right=2.54cm,top=2.54cm,bottom=2.54cm]{geometry}

\author{Leonard Hackel, Jochen Jacobs, Niklas Schelten}
\title{Manual zur Benutzung des 3D-Druckers}
\begin{document}
\maketitle
\begin{center}
\includegraphics[scale=0.75]{res/31_1.jpg}
\end{center}
\pagebreak
\tableofcontents
\pagebreak
%---------------------------Jochen----------------------------------------------
\section{Erstellen von 3D Objekten}
Fuer die Erstellung von 3d-Objekten ist die Verwendung von spezieller 3d-Software notwendig. Welche Software am sinnvollsten ist, h"angt von dem zu erstellenden Objekt ab. Organische Objekte werden am Besten mit einem \enquote{normalen} 3d-Programm wie \textbf{Blender} erstellt, W"ahrend technische Objekte wie Zahnr"ader mit einem \textit{computer-aided design}-Programm (kurz CAD-Programm) erstellt - hier empfehle ich das kostenlose Programm \textbf{FreeCAD}. Beide Programme k"onnen Modelle als stl-Datei speichern - Dieses Format ist im 3d-Druck "ublich. Im Gegensatz zur herk"omlichen 3d-Modelierung muss bei der Erstellung von zu druckenden Objekten nicht versucht werden die Anzahl der Poligone moeglichst gering zu halten - im Gegenteil kann eine gr"o"sere Anzahl an Poligonen nie schaden: Der 3d-Drucker beachtet die in der 3d-Graphik h"aufig benutzten Normalen nicht - dadurch erscheinen gedruckte Objekte sehr viel weniger Rund als sie auf dem Bildschirm aussehen.

Die verwendung der beiden Programme ist sehr komplex und kann hier nicht komplett beschrieben werden. Besonders zu empfehlen sind gerade in diesem Bereich auch Toutorial-Videos auf YouTube. Als Beispiel zur erstellung habe ich zu beiden Programmen je ein kurzes Video erstellt und auf YouTube hochgeladen:
\begin{itemize}
  \item \textbf{Blender} \url{https://youtu.be/BRPZsZmmzIY}
  \item \textbf{FreeCAD} \url{https://youtu.be/2ebwCnLCgOo}
\end{itemize}

\subsection*{Downloads:}
\begin{itemize}
  \item \textbf{Blender} \url{http://www.blender.org/download/}
  \item \textbf{FreeCAD} \url{http://www.freecadweb.org/}
\end{itemize}
\pagebreak
%---------------------------Jochen----------------------------------------------
\section{Der Druck}
% neuen Drucker hinzufügen/vor der ersten Nutzung
Da Cura den RF1000 nicht als Standard dabei hat, muss dieser vorher als neuer Drucker hinzugefügt werden. Dazu muss man unter \textit{Machine}, \textit{Add new machine...} dieser neu erstellt werden. Da er keine Voreinstellung ist, muss im \textit{Configuration Wizard} \textit{Other} gewählt werden, ebenso wie unter \textit{Other machine information} \textit{Custom...} ausgewählt sein muss. Nun kann man Namen (\textit{Machine name}), also RF1000, und Größe unter \textit{width}, \textit{depth} und \textit{height} einstellen (für den RF1000 245,235,200 mm in dieser Reihenfolge). \textit{Nozzle Size} ist standardmäßig 0.5 und für den RF1000 korrekt. Zusätzlich muss die Option \textit{Heated Bed} ausgewählt werden, die Option \textit{Bed center [...]} ist nicht zutreffend.\\
Danach müssen die anderen Einstellungen, welche sich auf der SD-Karte finden lassen, geladen werden. Dazu muss unter \textit{File}, \textit{Open Profile...} die Datei \textit{Settings.ini} geladen werden. Danach ist der RF1000 als Drucker eingestellt und kann für den Druck konfiguriert werden.\\
\includegraphics[scale=0.4]{res/Cura-window.png}\\

Um nun in diese Ansicht zu gelangen, muss man bei Cura im Reiter \textit{Expert} unter \textit{Switch to full settings} die Ansicht ändern. Dabei verschwinden die Schnelleinstellungen und diese Ansicht erscheint. Das hat den Vorteil, dass die Einstellungen viel genauer vorgenommen werden können.\\
Das Vorschaufenster bei Cura besteht aus zwei großen Teilen. Auf der linken Seite sind die Einstellungen, auf der rechten ist eine 3D-Ansicht des Objektes. Dabei sind verschiedene Strecken in verschiedenen Farben angezeigt. Innenstrecken sind dabei grün, Druckstrecken, die später Außenflächen ergeben, sind rot, und Strecken, welche der Druckkopf zurücklegt, dabei aber nicht druckt, sind dünner und blau. In gelb dargestellt sind diejenigen Strecken, die später Füllung sind.\\
Die meisten Einstellungen sind selbsterklärend. So gibt die \textit{Layer heigt} an, wie hoch die einzelnen Drucklagen sind, wodurch damit auch die Anzahl der Lagen bestimmt wird. \textit{Shell thickness} gibt an, wie viele Spuren außen gedruckt werden, bevor der Bereich als Innenraum gilt und nur zu einem bestimmten Prozentsatz gefüllt wird. dieser ist unter \textit{Fill Density} einstellbar. So wie \textit{Shell thickness} die Dicke der seitlichen Außenhaut angibt, gibt \textit{Bottom/Top thickness} dies für die oberen und unteren Schichten an.\\
\textit{Print speed} gibt an, wie schnell sich der Druckkopf relativ zu der Heizplatte bewegt, wenn er druckt (unter \textit{Advaced} gibt es noch andere Bewegungsarten). Hierbei gilt generell, dass eine geringere Geschwindigkeit die Qualität, allerdings auch die Druckzeit erhöht, wodurch ein geeignetes Mittelmaß zu finden ist. \textit{Printing temperature} und \textit{Bed temperature} sind von dem Druckmaterial abhängig, für PLA empfiehlt sich 210/60. Generell empfehlen wir die oben gezeigten Werte für die bisher genannten Einstellungen bei einem Druck mit PLA.\\
Unter \textit{Support} finden sich Einstellungen, die den Drucker bei schwereren Aufgaben wie Überhängen unterstützen. Sollte das Objekt Überhänge haben, empfehlen wir den \textit{Support type} \textit{Touching buildplate}, wenn es sich aber nur um kleine oder steile Überhänge handelt, ist ein Support nicht empfehlenswert (ab 60\% ist Support benötigt, ansonsten muss von Fall zu Fall unterschieden werden). Als \textit{Platform adhesion type} empfehlen wir einen \textit{Brim} der Breite 10, damit sich der Drucker \enquote{eindrucken} kann, wodurch Filamentaussetzer und Stotterer vermieden werden. Bei Objekten, welche leicht umfallen oder wackeln, ist ein \textit{Raft} zu empfehlen, da es den Druck stabilisiert. Allerdings kostet der Druck eines solchen viel Zeit.\\
\textit{Filament}-Settings sollten bei 3 (\textit{Diameter}) und 100(\textit{Flow}) gelassen werden.
\pagebreak
%---------------------------Niklas----------------------------------------------
\section{Probleme beim Drucken}
\iffalse
Probleme:
-Aus Extruder kommt nichts raus
-Druck hält nicht
-Extruder kratzt auf der Heizplatte
-Filament bricht
-Extruder verstopft (sollte bei Beachtung der anderen Fehler nicht passiere)
\fi

Beim Drucken von 3D Objekten können viele verschieden und unterschiedlich schwerwiegende Probleme auftreten. Hier sind die jeningen gelistet, dioe uns am häufigsten passiert sind:\\
\begin{description}

\item \textbf{Aus dem Extruder kommt nichts heraus.}\\
\addcontentsline{toc}{subsection}{Aus dem Extruder kommt nichts heraus}
Für dieses Problem kann es verschiedene Ursachen geben:
\begin{itemize}
\item \textit{Es ist kein Filament eingelegt:}\\
In diesem Fall ist der Extruder auf die für das gewünschte Filament benötigte Temperatur (PLA $210^\circ$C) zu erhitzen. Wenn dieser heiß ist kann man das Filament in den Extruder einführen und den Extrudervorschub manuell erhöhen bis unten Filament heraus tropft. Dies kann lange dauern, wenn der Extruder vorher leer war.\\
\textbf{ACHTUNG: Bei zu schnellem Vorschieben kann es passieren, dass das Filament sich nicht schnell genug erhitzt und dann den Extruder verstopft.}

\item \textit{Es ist vorher Filament aus dem Extruder gelaufen, ohne, dass welches nachgechoben wurde:}\\
In diesem Fall muss der Extruder auf die für das gewünschte Filament benötigte Temperatur (PLA: $210^\circ$C) erhitzt werden. Wenn dieser heiß ist kann man den Extrudervorschub manuell erhöhen bis unten Filament heraus tropft. Dies kann lange dauern, wenn der Extruder vorher leer war.\\
\textbf{ACHTUNG: Bei zu schnellem Vorschieben kann es passieren, dass das Filament sich nicht schnell genug erhitzt und dann den Extruder verstopft.}
\end{itemize}
\vspace{10pt}

\item \textbf{Das gedruckte Objekt hält nicht auf der Heizplatte.}\\
\addcontentsline{toc}{subsection}{Das gedruckte Objekt hält nicht auf der Heizplatte}
Für dieses Problem gibt es keine ultimative Lösung. Wir haben haben mehrere Faktoren gefunden, die dafür verantwortlich sind:
\begin{itemize}
\item \textit{Temperatur der Heizplatte}\\
Die Temperatur der Heizplatte ist zumindest für PLA auf $60^\circ$ zu erhitzen.
\item \textit{Beschaffenheit des Objekts}\\
Je nach dem, ob das Objekt eine kleine Standfläche hat und hoch ist, bzw. eine sehr breite Standfläche hält es unterschiedlich gut. Weiterhin ist es wichtig einen \enquote{Brim} oder einen \enquote{Raft} zu drucken. Ein \enquote{Brim} sind mehrere Linien um das objekt herum, die schnell zu drucken sind und den Halt der Objektes verbessern. Ein \enquote{Raft} hingegen ist eine Platform, auf der das Objekt gedruckt wird. Diese hält durch eine besondere Beschaffenheit an der Unterseite besonders gut auf dem Heizbrett, dauert aber auch entsprechend lange zu drucken.
\item \textit{Position auf dem Heizbrett}\\
Wir haben festgestellt, dass Objekte die auf der linken , hinteren Ecke gedruckt werden besser halten als andere.
\item \textit{Der Extruder ist zu hoch über der Heizplatte}\\
Wenn der Extruder zu hoch über der Heizplatte druckt, kühlt das Filament ab, bevor es auf die Heizplatte trifft und hält dort nicht mehr. Daher sollte man \textbf{VORSICHTIG} den Extruder über die Knöpfe herunter fahren.
\item \textit{Wenn nichts hilft}\\
Sonst haben wir in einem Forum gelesen, dass Hairspray (möglichst das billigste) auf der Heizplatte den Halt stark verbessert. Momentan liegt ein Haarspray bei dem 3D-Drucker dabei und hat auch schon seine Dienste geleistet.
\end{itemize}
\vspace{10pt}

\item \textbf{Der Extruder kratzt auf der Heizplatte.}\\
\addcontentsline{toc}{subsection}{Der Extruder kratzt auf der Heizplatte}
Sollte der Extruder auf der Heizplatte kratzen sollte der Druck direkt pausiert oder abgebrochen werden um Schaden an der Heizplatte zu verhindern. Wenn auf der Heizplatte leichte schwarze Spuren zu erkennen sind sollten diese im kalten Zustand der Heizplatte so weit wie möglich abgewischt werden. Das Problem liegt darin, dass die Heizplatte falsch gescannt wurde. Dies sollte dann nachgeholt werden. Auf dem 3D-Drucker im Menü \textit{Configuration/Z Calib./Heat Bed Scan} (siehe offizielles Manual S.57 ff.).\\
Temporär kann man dies auch über die manuelle Kontrolle des Extruders beheben, wobei man unbedingt aufpassen muss, dass der Extruder nicht die Platte zerstört, oder die Platte den Z-Schalter.
\vspace{10pt}

\item \textbf{Das Filament bricht.}\\
\addcontentsline{toc}{subsection}{Das Filament bricht}
Wenn das Filament bricht ist es am sinnvollsten, das Filament direkt über dem Extruder, also zwischen Extruder und Z-Moter abzuschneiden und dann neues nachzufüllen. \textbf{ACHTUNG: dazu muss der Extruder auf die für das gewünschte Filament benötigte Temperatur (PLA: $210^\circ$C) erhitzt werden.} Danach sollte noch ein bisschen Filament nachgedrückt werden um zu gewährleisten, dass das neue Filament auch in den Extruder eingefüllt wird.
\vspace{10pt}

\item \textbf{Der Extruder verstopft.}\\
\addcontentsline{toc}{subsection}{Der Extruder verstopft}
Jetzt weiß man auf jeden Fall, dass man einen der vorherigen Beschreibungen nicht richtig beachtet hat. Eigentlich ist der Extruder jetzt auszuwechseln, da dies aber kostspielig ist (\url{http://goo.gl/J6Bhq2} (Conrad.de 50\euro)) empfehlen wir zuerst zu versuchen, das Filament \textit{rauszuschmelzen}. Hierzu haben wir den Heizluft Föhn der Schule verwendet und den oberen Teil des Extruders so lange erhitzt, bis das Filament auf rausgelaufen ist und keines merh drin war. Hierbei muss man aber daruaf achten, dass die Kabel nicht schmelzen bzw. der Extruder anderweitig beschädigt wird. Andere Leute haben in Foreneinträgen berichtet, dass Filament aus dem Extruder \textit{rausgebohrt} zu haben. Dies haben wir uns nciht getraut, da wir keinen Milimeter genauen Bohrer zur Hand hatten.
\end{description}
\end{document}